% !TEX program = xelatex
\documentclass[10pt, a4paper]{article}

% --------------------------------------------------------------------
% การตั้งค่าแพ็กเกจและฟอนต์ (สำคัญมากสำหรับ XeLaTeX)
% --------------------------------------------------------------------

% แพ็กเกจหลักสำหรับจัดการฟอนต์ใน XeLaTeX
\usepackage{fontspec}

% แพ็กเกจสำหรับจัดการอักษรจีน ญี่ปุ่น เกาหลี
\usepackage{xeCJK}

% แพ็กเกจสำหรับปรับระยะขอบกระดาษ
\usepackage[a4paper, top=2cm, bottom=1cm, left=3cm, right=2cm]{geometry}

% แพ็กเกจสำหรับสร้างเส้นคั่น
\usepackage{linegoal}

% ตั้งค่าให้มีระยะห่างระหว่างย่อหน้า แทนการย่อหน้าแรก
\usepackage{parskip}

% แพ็กเกจสำหรับการจัดคอลัมน์ (ถ้าจำเป็น)
% ในที่นี้ใช้สำหรับจัดเนื้อหาเป็น 3 คอลัมน์
\usepackage{multicol}

% ตั้งค่าระยะห่างระหว่างคอลัมน์
\setlength{\columnsep}{1cm}

% ตั้งค่าระยะห่างระหว่างย่อหน้าภายในคอลัมน์
\setlength{\multicolsep}{0pt}

% --------------------------------------------------------------------
% การเลือกฟอนต์ (กรุณาตรวจสอบว่าคุณมีฟอนต์เหล่านี้ในเครื่อง)
% --------------------------------------------------------------------

% ตั้งค่าฟอนต์หลักสำหรับภาษาอังกฤษและตัวเลข
\setmainfont{Arial}

% ตั้งค่าฟอนต์สำหรับภาษาจีน
% หากคุณใช้ Windows อาจเปลี่ยนเป็น "SimSun" หรือ "Microsoft YaHei"
% หากคุณใช้ macOS อาจเปลี่ยนเป็น "PingFang SC" หรือ "Heiti SC"
\setCJKmainfont{仿宋}[Scale=1]  % กำหนดขนาด ตัวอักษร 80%

% สร้างคำสั่งสำหรับเรียกใช้ฟอนต์ไทยโดยเฉพาะ
% หากคุณใช้ Windows อาจเปลี่ยนเป็น "Leelawadee UI"
% หากคุณใช้ macOS อาจเปลี่ยนเป็น "Thonburi"
\newfontfamily{\thaifont}{TH SarabunPSK}[Scale=1]  % กำหนดขนาด ตัวอักษร 80%


% --------------------------------------------------------------------
% เนื้อหาเอกสาร
% --------------------------------------------------------------------

\title{般若波罗蜜多心经\\
    {\thaifont แปลบทปรัชญาปรมามิตตา (หฤทัยสูตร)}
    }
\author{}
\date{{\thaifont 14 กรกฎาคม พ.ศ. 2563}} 

\begin{document}

\maketitle

\begin{center}
\rule{\linewidth}{0.4pt}
\end{center}

% เว้นบรรทัด
\vspace{1cm}

% เริ่มเนื้อหาพระสูตร
% ใช้ \thaifont{} ครอบส่วนที่เป็นภาษาไทย

% เริ่มคอลัมน์ 3 คอลัมน์
\begin{multicols}{3}

观自在菩萨 \\
{\thaifont ครั้งนั้นเมื่อพระโพธิสัตว์ได้เจริญญาณทัศนะ}

行深般若波罗蜜多时 \\
{\thaifont เจริญปัญญาในเบื้องลึก}

照见五蕴皆空 度一切 苦厄 \\
{\thaifont เห็นซึ่งความว่างในขันธ์ทั้ง 5 พบเห็นทุกข์ทั้งมวล}

舍利子 \\
{\thaifont สารีบุตร}

色不异空 空不异色 \\
{\thaifont รูปไม่ต่างจากความว่าง ความว่างก็ไม่ต่างจากรูป}

色即是空 空即是色 \\
{\thaifont รูปก็คือว่าง ว่างก็คือรูป}

受想 行识 亦复如是 \\
{\thaifont เวทนา สัญญา สังขาร วิญญาณต่างก็เป็นเช่นนี้}

舍利子 \\
{\thaifont สารีบุตร}

是诸法空相 \\
{\thaifont คือความว่างจากความปรุงแต่งทั้งมวล}

不生不灭 \\
{\thaifont ไม่มีการเกิด ไม่มีการดับ}

不垢不净 \\
{\thaifont ไม่มีสกปรก ไม่มีใสสะอาด}

不增不减 \\
{\thaifont ไม่มีเพิ่ม ไม่มีลด}

是故空中无色 \\
{\thaifont ก็เมื่อตัดแล้วซึ่งรูป}

无受想行识 \\
{\thaifont ไม่มีเวทนา สัญญา สังขาร วิญญาณ}

\columnbreak

无眼耳鼻舌身意 \\
{\thaifont ไม่มีทั้ง ตาหูจมูกลิ้นกายใจ}

无色声香味触 法 \\
{\thaifont ก็ไม่มีความปรุงแต่งรูป รส กลิ่น เสียง สัมผัส}

无眼界 乃至无意识界 \\
{\thaifont เห็นทั่วถึงโลก จึงรู้ทั่วถึงโลก}

无无明 亦无无明尽 \\
{\thaifont จบซึ่งอวิชชา จบความไม่รู้ทั้งมวลสิ้น}

乃至无 老死 亦无老死尽 \\
{\thaifont จึงจบซึ่งความแก่ ความตาย จบความแก่และความตายสิ้น}

无苦集灭道 \\
{\thaifont จบสิ้นซึ่ง ทุกข์ สมุหทัย นิโรธ มรรค}

无智亦无得 \\
{\thaifont ทั้งไม่คิดสรรค์ทั้งไม่ไม่ปรุงแต่ง}

以无所得故 \\
{\thaifont เป็นสิ่งที่ได้จากความว่างเช่นนี้เอง}

菩提萨陲 \\
{\thaifont พระโพธิสัตว์นั้น}

依般若波罗蜜多故 \\
{\thaifont เมื่อเจริญปรัชญาปรมามิตตาแล้ว}

心无挂碍 无挂碍故 \\
{\thaifont จิตปราศจากนิวรณ์ทั้งหมดทั้งสิ้น}

无有恐怖 \\
{\thaifont จิตจึงไม่มีความหวาดกลัว ไม่หวั่นไหว}

远离颠倒梦想 \\
{\thaifont ห่างไกลความคิดเพ้อฝัน}

究竟涅盘 \\
{\thaifont เข้าสู่พระนิพพาน}

\columnbreak

三世诸佛 \\
{\thaifont พระพุทธเจ้าทั้งสามโลกนั้น (อดีต ปัจจุบัน อนาคต)}

依般若波罗蜜多故 \\
{\thaifont ก็เนื่องจากเจริญปรัชญาปรมามิตตานี้แล้ว}

得阿耨多罗三藐三菩提 \\
{\thaifont จึงเห็นชัดในญาณทั้งสามของพระโพธิสัตว์}

故知般若波罗蜜多 \\
{\thaifont พึงรู้เถิดว่าปรัชญาปรมามิตตานี้}

是大神咒 \\
{\thaifont เป็นมนต์วิเศษ}

是大明咒 \\
{\thaifont เป็นมนต์สว่าง}

是无上咒 \\
{\thaifont ไม่มีมนต์ใดยิ่งกว่า}

是无 等等咒 \\
{\thaifont ไม่มีมนต์ใดเทียบได้}

能除一切苦 真实不虚 \\
{\thaifont สามารถนำให้พ้นทุกข์ทั้งมวลได้แท้จริง}

故说般若波罗蜜多咒即说咒曰 \\
{\thaifont บทปรัชญาปรมามิตตานี้และเนื้อหาแห่งมนต์}

揭谛揭谛 波罗揭谛 \\
{\thaifont มาเถิด มาเถิด ท่านจงมาดูเถิด}

波罗僧揭谛 \\
{\thaifont จงมาถึงบทอันเกษม}

菩提娑婆诃 \\
{\thaifont กระแสแห่งโพธิสัตว์}

\end{multicols}

\vspace{1cm}
\hfill{\thaifont โดย} 罗宗保 {\thaifont อธิปัตย์ ล้อวงศ์งาม}

\end{document}